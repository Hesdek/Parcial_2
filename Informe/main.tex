\documentclass{article}
\usepackage[utf8]{inputenc}
\usepackage{graphicx}

\title{Informe Parcial II}
\author{daniel.perez19 }
\date{September 2021}

\begin{document}

\begin{titlepage}
    \begin{center}
        \vspace*{1cm}
            
        \Huge
        \textbf{Informe Parcial 2}
            
        \vspace{0.5cm}
        \LARGE
        Informática II
            
        \vspace{1.5cm}
            
        \textbf{Daniel Perez Gallego CC. 1193088770\\Jorge Montaña Cisneros CC.  1007327968}
            
        \vfill
            
        \vspace{0.8cm}
            
        \Large
        Departamento de Ingeniería Electrónica y Telecomunicaciones\\
        Universidad de Antioquia\\
        Medellín\\
        Septiembre de 2021
            
    \end{center}
\end{titlepage}

\tableofcontents

\section{Análisis}
\subsection{Análisis del problema}
Mientras más pequeña sea la matriz de LEDs, menos información deberemos exportar, será más eficiente y fácil, sin embargo, la imagen se volverá dificil de reconocer para el usuario, por lo tanto, acordamos hacer la matriz de LEDs de 32x32\\
\\Pensamos a forma de solución para el submuestreo, separar las filas y columnas pares, dejando solamente las columnas pares, de este modo, tendremos la misma imagen, pero recortada a la mitad, luego realizamos el mismo proceso pero cortando las filas, obteniendo un tamaño menor pero proporcional a la imagen original, repiendo el proceso hasta obtener el tamaño deseado.\\
\includegraphics[width=4cm]{Imagenes/recorte1.jpeg}
\includegraphics[width=4cm]{Imagenes/recorte2.jpeg}
\includegraphics[width=4cm]{Imagenes/recorte3.jpeg}


\subsection{Tareas a realizar}
1. Realizar una función para sacar todos los valores separados del RGB.\\
2. Aplicar la técnica de "Recortar la imagen" analizada antes e implementarla a una clase "Imagen".\\
3. Segunda función será la altura en pixeles de nuestra imagen. Buscamos mas documentación para comprender más sobre el tema tratado.\\


\end{document}
